\documentclass[11pt]{article}
\usepackage[utf8]{inputenc}	% Para caracteres en español
\usepackage{amsmath,amsthm,amsfonts,amssymb,amscd,multirow,booktabs,fullpage,lastpage,enumitem,fancyhdr,mathrsfs,wrapfig,setspace,calc,multicol,cancel,amsmath,empheq,framed,xcolor}
\usepackage[retainorgcmds]{IEEEtrantools}
\usepackage[margin=3cm]{geometry}
\newlength{\tabcont}
\setlength{\parindent}{0.0in}
\setlength{\parskip}{0.05in}
\usepackage[most]{tcolorbox}
\colorlet{shadecolor}{orange!15}
\parindent 0in
\parskip 12pt
\geometry{margin=1in, headsep=0.25in}
\theoremstyle{definition}
\newtheorem{defn}{Definition}
\newtheorem{reg}{Rule}
\newtheorem{exer}{Exercise}
\newtheorem{note}{Note}
%%%%%%%%%%%%%%%%%%%%%%%%%%%%%%%%%%%%%%%
\begin{document}
\setcounter{section}{0}
\title{Apuntes}	
\thispagestyle{empty}	
\begin{center}
	{\LARGE \bf Apuntes}\\
	Rojas C. Aarón\\
	Semestre\\
\end{center}
%%%%%%%%%%%%%%%%%%%%%%%%%%%%%%%%%%%%%%%%
\section{Sección 1}
\subsection{Subsección 1}
Lorem ipsum dolor sit amet, consectetur adipiscing elit. Aliquam porta euismod neque vitae imperdiet. Suspendisse ac est condimentum, pellentesque dui a, efficitur nibh. 
	\begin{note}
	\textbf{Lorem ipsum dolor sit amet, consectetur adipiscing elit. Aliquam porta euismod neque vitae imperdiet. Suspendisse ac est condimentum, pellentesque dui a, efficitur nibh.}
	\end{note}
Lorem ipsum dolor sit amet, consectetur adipiscing elit. Aliquam porta euismod neque vitae imperdiet. Suspendisse ac est condimentum, pellentesque dui a, efficitur nibh.
$$E=mc^{2}$$
Lorem ipsum dolor sit amet, consectetur adipiscing elit. Aliquam porta euismod neque vitae imperdiet. Suspendisse ac est condimentum, pellentesque dui a, efficitur nibh.
\subsection{Subsección 2}
	\begin{shaded}
	\textbf{Example ecuation} \newline
		\begin{equation}
		F_{tide} = -GM_mm(\frac{\hat{d}}{d^2}-\frac{\hat{d_0}}{d_0^2})
		\end{equation}
	Where:
		\begin{equation*}
			\begin{split}
			G = \text{Gravitational Constant} \\
			d = \text{Object's Position Relative to Moon} \\
			d_0 = \text{Earth's Center Relative to the moon}\\
			M_m = \text{Mass of the moon}
			\end{split}
		\end{equation*}
	\end{shaded}
\newpage
\subsection{The Angular Velocity Vector}
The rest of the notes and the chapter will over reference frames that are rotating with respect to the inertial reference frame, so angular velocity has to be used. 
\begin{defn}
\textbf{Euler's Theorem} - The most general motion of any body relative to a fixed point \textit{O} is a rotation about some axis through \textit{O} To specify this rotation about a given point O, we only have to give the direction of the axis and the rate of rotation, or angular velocity $\omega$. Because this has a magnitude and direction, it is an obvious choice to write this rotation vector as $\omega$, the angular velocity vector. That is:
\begin{equation}
\omega = \omega\textbf{u}
\end{equation}
Where \textbf{u} is the unit vector
\end{defn}
%%%%%%%%%%%%%%%%%%%%%%%%%%%%%%%%%%%%%%%%
\end{document}
